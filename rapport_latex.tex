\documentclass[10pt,a4paper]{report}
\usepackage[utf8]{inputenc}
\usepackage[francais]{babel}
\usepackage{colorx}
\usepackage{graphicx}
\usepackage{amsmath,amsfonts,amssymb}


\begin{document}

\title{Partie LateX}


\tableofcontents

\section{La grammaire}
Pour la partie LaTeX, nous avons conçu une grammaire. Elle débute au moment où l'on rencontre un {\bf begin\{document\}} dans le fichier tex. Alors on parcourt le BODY dans lequel on retrouve tout ce qui est parsé dans le fichier latex. Certains cas tels que l'appel de listes, de tableaux et d'autres appelent des règles qui leur sont propres. Le parcours de cette grammaire se finit dès que l'on rencontre un {\bf end\{document\}}.

\section{Les commandes de présentation}

\subsection{Gras, italique, sur-ligné}

\texttt{Gras}, \textit{italique} et \underline{sur-ligné} fonctionnent.
On aurait pu rajouter le sur-lignage, mais ceci demandé encore à l'utilisateur d'utiliser une nouvelle commande avec des paramètres (ce qui ne fonctionne pas ici).

\subsection{La couleur du texte}

Les couleurs reconnues sont : \textcolor{black}{black}, \textcolor{white}{white}, \textcolor{red}{red}, \textcolor{green}{green}, \textcolor{blue}{blue}, \textcolor{yellow}{yellow}, \textcolor{cyan}{cyan}, \textcolor{magenta}{magenta}, \textcolor{brown}{brown}, \textcolor{orange}{orange}, \textcolor{violet}{violet}, \textcolor{purple}{purple}, \textcolor{gray}{gray}, \textcolor{lightgray}{lightgray} & \textcolor{darkgray}{darkgray}\\Pour les utiliser depuis un fichier tex, il faut penser à rajouter le package xcolor.


\section{Les environnements de base}

\subsection{Les listes}

Pour convertir les \underline{listes} en html, on envoie un {\bf token} afin d'arriver dans la règle {\bf begin_itemize} de la grammaire, nous permettant ainsi d'inclure des balises html correspondant aux listes et son corps qui se compose des items et de la règle body de la grammaire. Cela nous permet de pouvoir imbriquer différentes balises dans nos structures, une liste numérotée de tableaux par exemple.
\begin{itemize}
\item premier item
\item second item
\end{itemize}

\subsection{Les listes numérotées}

Pour les \underline{listes numérotées}, le fonctionnement est le même que pour les listes à la différence que nous n'utilisons pas la même balise html pour les retranscrire.
\begin{enumerate}
\item enum1
\item enum2
\end{enumerate}

\subsection{Les tableaux}

\begin{tabular}{lccr}
Mon & Tableau & semble & être fonctionnel \\
$1$ & $1+2$ & $1+2+3$ & $1+2+3+4$\\
\end{tabular}

\begin{tabular}{rcl}
\begin{tabular}{cc}
T & a \\
\end{tabular}
& b &l \\
\begin{tabular}{cc}
E & a \\
\end{tabular}
& u &x \\
\end{tabular}

Les tableaux sont créés avec l'aide des variables param_tabular et index_param_tabular. On aurait pu implémenter plus "proprement", mais le temps nous a manqué.

\section{Le mécanisme des sections}

\subsection{Les sections}

Les \underline{sections} sont tous les titres que vous voyez sur notre page Html. Pour avoir ces titres depuis un fichier tex, on récupère ce qui se trouve entre les accolades de la commande section, subsection, subsubsection, ainsi que leurs homologues avec une étoile. pour ce faire, nous avons copié le contenu de {\bf yytext} et l'avons stocké dans une variable tout en parcourant le Body de notre grammaire.

\subsection{La table des matières}

La \underline{table des matières} était une difficulté à surmonter. Nous avions pensé à plusieurs approches. Tout d'abord la mettre en fin de fichier, étant donné que nous n'avons pas tous les titres avant d'arriver à la fin du fichier. Nous avons ensuite pensé à faire deux parcours du fichier tex : un pour récupérer les titres, et l'autre pour créer la table des matières en début de page. Nous avons finalement opté pour une fonction javascript nous permettant de récupérer tous les titres html et ainsi de créer une table des matières à l'endroit où on le souhaite.

\section{Les liens internes}

Ceux-ci correspondent aux liens permettant de naviguer dans celui-ci, et non vers d'autres fichiers. En laTeX, on utilise des {\bf label} et des {\bf label*} pour placer une ancre, et des {\bf ref} pour retourner à l'emplacement de cette ancre. La retranscription de cette fonctionnalité est gérée dan la partie Lex du projet. On récupère l'id du label ainsi que le texte à afficher (sauf si c'est un label*). On fait la même opération pour le ref, à la différence que l'id de la réference sera l'ancre accrochée au label que la commande a spécifié. Comme ceci :
\ref{1}{Remontez}

\section{La citation de code}

Une \underline{citation de code} marche de la manière suivante : une fois la commande {\bf citecode} reconnue avec ses deux paramètres(qui sont des liens), une fonction Javascript permet de créer une balise html englobant tout ce qui se trouve entre ces deux liens, et nous copions ensuite le contenu de cette balise à l'endroit souhaité. Nous aurions voulu faire marcher cette fonctionnalité en utilisant du javascript car c'était à nos yeux la seule façon de procéder. Cependant, notre niveau en JavaScript ainsi que notre manque de temps ne nous a pas permis de finaliser cette fonctionnalité.

\section{Les formules mathématiques}

Dans ce projet, on fait la distinction entre 2 types de formules mathématiques:
\begin{enumerate}
\item Les formules de type MathML -- \$ qui s'intégrent directement au texte
\item Les formules de type equation -- \$\$ qui sont centrées par rapport au texte
\end{enumerate}

Dans l'environnement de MathML, les indices $u_{n+1}$, les exposants $u^{n+1}$, les fractions $\frac{u+1}{v+2}$ et les racines $\sqrt{8+\sqrt{u+1}} +3$ sont fonctionnelles.
\\De plus, de nombreux symboles ont été implémentés:
\begin{enumerate}
\item Operateur : $ \left( \right) - \times \div \neq \equiv \sim \approx < \ll \leq > \gg \geq \pm \cdot \cdots : % + =  \in \notin  $
\item Symboles Logiques : $|  /  \neg \wedge \vee \oplus \Rightarrow \Leftarrow \Leftrightarrow \exists \forall \& \cap \cup \supset \subset $
\item Autres Symboles : $\prime \lfloor \rfloor \infty \emptyset  $
\end{enumerate}


On notera que les équations et les formules MathML auraient pû être codées sous le même format MathML, car il existe une option 'display' permettant de choisir de rendre l'expression mathématique dans son propre paragraphe <math display='block'> ou sur la même ligne que le texte ambiant <math display='inline'>.

\section{Les nouvelles commandes et les nouveaux environnements}

Les nouvelles commandes et les nouveaux environnements sont enregistrés dans des fichiers qui leurs sont propres, empilés afin de conserver tout les fichiers qui ont été créés, puis sont ensuites supprimés à la fin du parse.\\On a fait le choix d'une pile car celle-ci été déjà implémentée, mais on aurait pû utliser d'autres structures de données.
Toutes les commandes sans arguments optionnels fonctionnent, mais pas celles avec arguments.\\En effet, nous avons rencontrés des problèmes au niveau du parser, quand on a cherché à reconnaitre les expressions de type  \textbackslash\textbackslash"[[:alpha:]+(``{``[[:alnum:]]+''}'')+ .

\end{document}
