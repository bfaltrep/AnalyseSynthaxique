\documentclass[10pt,a4paper]{report}
\usepackage[utf8]{inputenc}
\usepackage[francais]{babel}
\usepackage{xcolor}
\usepackage{graphicx}
\usepackage{amsmath,amsfonts,amssymb}


\begin{document}

\textcolor{blue}{Piou} \textcolor{red}{Piiou} !

\date{\today}

\section{Mon premier est:}        

Test Caractères spéciaux : [ Crochets ]
\begin{enumerate}
\item \textbackslash o/ \textbackslash o/ \textbackslash o/ \textbackslash o/
\item J'ai oublié de tester \textit{ça} \textbackslash\textbackslash
\item \underline{Ben} \& Jerry's
\item Money ! \$\_\$
\item \{Accolades\}
\item [Crochets]
\item test !
\end{enumerate}

\subsection{Mon second ressemble à:}

%Je suis un commentaire !

On pourra se rappeler qu'on aurait pu faire tout les maths avec MathMl, car il y a une "option" qui permet de choisir entre en paragraphe ("block") ou incrusté dans le texte ("inline").

$ 2^n $
$ (3x - 2y)^2 $
$ 8^{8x \div 4} $
$ {yo}_n $
$ u_{coucou} $
$ {{bonjour}_{ca va}}^{tout baigne} $
\\
$ \frac{8}{x} $
$ 78+y+\infty $
$ \frac{5+x}{3+1} $

Je suis un essai \textasciitilde n de texte !
Un retour à la ligne.




Mais quand vais je retourner à \textasciicircum la ligne\\ ou faire un alinéa ?\\
Ici ? Test\newline Test2
Test

Ici ?
\begin{equation}
\label{prout}
x = 2
\end{equation}

Dans l'equation~\eqref{prout}, blabla

\begin{equation*}
\label{eq:2}
x = 2
\end{equation*}

Dans l'equation~\eqref{eq:2}, blabla \(f(x) = 0 -7x = 1 * 8 = 3 /4 \) j'ai fini cette equation qui sort de nulle part ! :)  C'est fou comme on revient toujours à la ligne !

Dans la partie Suivante, je vais tester TOUT mes caractères présents dans MathML que j'ai encodé: \( - \times \div \neq \equiv \sim \approx < \ll \leq > \gg \geq \pm \cdot \cdots \| : \% + = /
 \neg \wedge \vee \oplus \Rightarrow \Leftarrow \Leftrightarrow \exists \forall \&
\cap \cup \supset \subset \emptyset \in \notin
\prime \lfloor 18 \rfloor \infty
\)

Ou encore là ? \\Sauf quand on le précise :p

\begin{itemize}
\item PIOU PIOU
\begin{enumerate}
\item enum1
\item ENUM2
\end{enumerate}
\item PILPIL
\end{itemize}

POYPOY
\begin{enumerate}
\item enum3
\item ENUM4
\end{enumerate}
\begin{tabular}{rlc}
A droite ... & A gauche ... & Au centre ! \\
{\bf \textcolor{red}{Bob}} & 300 & 15 \\
{\itshape Chris} & 8 & \texttt{7800} \\
\underline{ \textcolor{blue}{0}} & \textcolor{white}{h} & \textcolor{red}{0} \\
\end{tabular}

\subsubsection{Mon tout est un:}
Je re-re-test tout ici !
\includegraphics{Mini_Amel.png}
JZHGEFIUZHJ

\begin{tabular}{cccc}
KJH & GZEF & O & IUZ \\
LZ & KE & & \\
L & JH & SDF & PIZE \\
\end{tabular}

\tableofcontents
\end{document}
