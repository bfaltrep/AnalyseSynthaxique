\documentclass[10pt,a4paper]{report}
\usepackage[utf8]{inputenc}
\usepackage[francais]{babel}
\usepackage{xcolor}
\usepackage{graphicx}
\usepackage{amsmath,amsfonts,amssymb}



\begin{document}

\title{Partie Doxygen}


\section{Description}

Les commentaires Doxygen servent à documenter un fichier C. La documentation est générée sous forme html. Elle a pour syntaxe :
\begin{verbatim}
/** (ou /!*)
 * \commande texte
 * \commande texte
 * texte
 * \commande texte
 */
\end{verbatim}

Les commandes Doxygen reconnues par notre projet sont : fn, brief, param et return. Une commande et son texte peuvent prendre une ou plusieurs ligne comme dans l'exemple ci-dessus.

La documentation est analysée, puis elle est générée en syntaxe html avec le css adéquat.

\section{Problèmes rencontrés}
Nous avons eu des problèmes sur la manière de reconnaître les différents motifs dans un commentaire doxygen car il n'y avait pas une ligne pour une commande. Finalement nous avons crée une "commande actuelle" qui se met à joue seulement quand il y a une nouvelle commande de lue.


\section{Améliorations possibles}
Une amélioration possible serait que l'analyseur récupère toutes les prototypes de fonction dans le .h et qu'il génère un début de commentaire doxygen avec les bonnes balises en fonction du nombre de paramètres de la fonction. L'utilisateur n'aurait plus qu'à rajouter son texte.

Une autre amélioration serait de gérer les balises-liens entres les descriptions de fonctions afin de faciliter le déplacement et la lecture de la documentation par l'utilisateur.

\end{document}
